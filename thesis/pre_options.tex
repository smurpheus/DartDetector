%!TEX root = ../thesis.tex

% TikZ
\usetikzlibrary{matrix,shapes,positioning,calc,fit}

% Captions
\captionsetup[table]{position=top}

% Umgebungen
\newtheorem*{example*}{Beispiel}
\newtheorem{example}{Beispiel}

%\newtheorem{example}{Beispiel}[chapter]
\newtheorem{definition}{Definition}[chapter]

% Format für Referenzen
\newrefformat{chap}{Kapitel~\ref{#1}} 
\newrefformat{sec}{Abschnitt~\ref{#1}} 
\newrefformat{subsec}{Abschnitt~\ref{#1}}
\newrefformat{app}{Anhang~\ref{#1}}
\newrefformat{fig}{Abbildung~\ref{#1}}
\newrefformat{tbl}{Tabelle~\ref{#1}}
\newrefformat{lst}{Listing~\ref{#1}}
\newrefformat{eql}{Ausdruck~\ref{#1}}
\newrefformat{algo}{Algorithmus~\ref{#1}}
\newrefformat{ex}{Beispiel~\ref{#1}}
\newrefformat{def}{Definition~\ref{#1}}
\newrefformat{lemma}{Lemma~\ref{#1}}

% Kopf- und Fußzeile
%i - innen, c - mitte, o - außen; head - Kopfzeile, foot - Fußzeile
\ihead{}
\chead{}
\ohead{\leftmark}
\ifoot{}
\cfoot{}
\ofoot{\pagemark}
\pagestyle{scrheadings} %eigenen Stil für alle Seiten setzen

\renewcommand{\sectionmark}[1]{\markboth{\thesection\ #1}{}}
\renewcommand{\subsectionmark}[1]{\markright{#1}}
\newcommand{\BVAnd}{\mathbin{\sqcap}}

% Symbole und Variablen
\makeatletter
\newcommand{\diamonds}{\ensuremath{\diamondsuit}}
\newcommand{\hearts}{\ensuremath{\heartsuit}}
\newcommand{\spades}{\ensuremath{\spadesuit}}
\newcommand{\clubs}{\ensuremath{\clubsuit}}
\newcommand{\cp}{\@ifstar{cp}{cards\_player}}
\newcommand{\cs}{\@ifstar{cs}{cards\_skat}}
\newcommand{\ac}{\@ifstar{ac}{available\_cards}}
\newcommand{\tp}{\@ifstar{tp}{turn\_player}}
\newcommand{\trumpd}{trump\_diamonds}
\newcommand{\trumph}{trump\_hearts}
\newcommand{\trumps}{trump\_spades}
\newcommand{\trumpc}{trump\_clubs}
\newcommand{\trumpg}{trump\_grand}
\newcommand{\mf}{\@ifstar{mf}{\mathit{must\_follow}}}
\newcommand{\trick}{\@ifstar{t}{trick}}
\newcommand{\points}{points}
\newcommand{\tpoints}{trick\_points}
\newcommand{\pdec}{\mathit{total\_points\_declarer}}
\newcommand{\pdef}{\mathit{total\_points\_defenders}}
\newcommand{\winning}{\mathit{winning}}
\newcommand{\follow}{\mathit{following}}
\newcommand{\wahr}{\mathit{wahr}}
\newcommand{\falsch}{\mathit{falsch}}
\makeatother

% Abstände
\renewcommand{\baselinestretch}{1.3} % Zeilenabstand
\setlength{\parskip}{1.0ex} % Abstand zwischen Absaetzen
\setlength{\parindent}{0ex} % keine Einrueckung am Anfang eines Absatzes

% Nummerierung
\setcounter{tocdepth}{2}
\setcounter{secnumdepth}{3}

% TikZ
\tikzset{square matrix/.style={
    matrix of nodes,
    column sep=-\pgflinewidth, row sep=-\pgflinewidth,
    nodes={draw,
      text height=#1/2+0.75ex,
      text depth=#1/2-0.75ex,
      text width=#1,
      align=center,
      inner sep=0pt,
      scale=0.8,
    },
  },
  square matrix/.default=1cm,
  index/.style={column 1/.style={nodes={draw=none}}},
  vindex/.style={row 1/.style={nodes={draw=none}}},
}