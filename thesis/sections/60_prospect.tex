%!TEX root = ../thesis.tex
\chapter{Ausblick und Fazit}
\label{chap:prospect}
Zum Abschluss wird eine Zusammenfassung der aufgetretenen Probleme gegeben. Zudem werden mögliche Lösungsansätze erläutert. Erweiternd wird darauf eingegangen in wieweit die in \prettyref{sec:motivation} genannten Ziele für diese Bachelorarbeit erfüllt werden konnten.

Aus den in \prettyref{sec:results} erlangten Erkentnissen ergeben sich folgende Probleme:
\begin{itemize}
	\item Verdeckung
	\item Nachbarfelder erkannt
	\item Darts berühren sich
\end{itemize}

\section*{Ausblick}
Eine Verdeckung erkennen zu können ist mit Software teils  realisierbar. So ist es grundsätzlich denkbar, dass partiell verdeckte Darts vervollständigt werden könnten. Vorausgesetzt vorher geworfene Darts wurden erkannt, wäre es denkbar die multiplen Konturen, die durch die Verdeckung entstehen, zusammenzuführen. Diese Möglichkeit ist allerdings auf die Fälle beschränkt, in denen ein Dart nur zu einem kleinen Teil verdeckt ist. Werden zu große Teile eines Darts verdeckt kann anhand der Informationen der einen Webcam keine Aussage über das erzielt Feld gegeben werden. 
Dieses Problem wäre vermeidbar, indem ein zweiter Kamera-Winkel hinzugefügt wird. 
So wäre es denkbar eine zweite Webcam aufzustellen, die von der entgegengesetzten Seite auf das Dartboard gerichtet ist. Mit zwei Webcams gäbe es zwei Möglichkeiten diese zu nutzen. 
Es ist denkbar die Kalibrierung der zweiten Kamera parallel zur ersten durchzuführen. So könnte eine zweite Ansicht eingebracht werden. 

Eine weitere denkbare Möglichkeit ist, die Kameras gemeinsam zu kalibrieren und eine Stereo-Sicht mit Tiefen-Informationen zu implementieren. Vorteilhaft daran wäre, dass erkannt werden könnte wie Darts ausgerichtet sind und anhand dessen eine Vermutung über die genaue Lage gemacht werden könnte.

Analog zu zwei Webcams gibt es die Möglichkeit eine Stereo-Kamera mit Tiefen-Sensoren zu verwenden. Beispiele dafür sind die "`Kinect"' von Microsoft, oder die Asus "`Xtion"', welche ein Tracking von Objekten in einem Raum ermöglichen.

Durch die Beschaffenheit der Background-Substraction treten bei sich berührenden Darts verfälschte Vordergrund-Masken auf. Dies entsteht, da ein geworfener Dart einen bereits im Dartboard befindlichen berührt und in Bewegung versetzt. Dadurch sind diese auf der Vordergrund-Maske sichtbar und verfälschen das Ergebnis.
Eine Möglichkeit dies zu verhindern, ist den Background-Substractor anzupassen. Ebenfalls denkbar ist, sich auf Tiefeninformationen zu stützen, um Darts auf dem Board zu identifizieren. 


Die häufige Erkennung eines Darts, im Nachbarfeld des tatsächlich erzielten, könnte bereits durch eine Kamera mit verbesserter Auflösung veringert werden. Betrachtet man die aufgenommenen Bilder, so sind die Spitzen der Darts oft nicht von der Umgebung zu unterscheiden, da das Bild zu unscharf ist. Denkbar ist es ebenfalls eine zweite Kamrea näher am Dartboard zu platzieren, um eine schärfere Ansicht zu erlangen. So könnte mit der ersten Kamera eine grundlegende Erkennung durchgeführt und mit der zweiten weitere Informationen zur Entscheidung erlangt werden. 

\section*{Fazit}
Abschließend betrachtet wurde eine Basis geschaffen, um eine Erkennung von Darts mit Hilfe einer Webcam zu verwirklichen. Dabei gilt es allerdings einige Barrieren zu überwinden. Mit der gegebenen Hardware wird es nur schwer Möglich sein, eine Erkennungsrate über $98\%$ zu erreichen. So ist es möglich auf dem Stand dieser Arbeit aufzusetzen und mit den vorgeschlagenen Verbesserungen eine Möglichkeit, zur Erfassung der eigenen Trainingswürfe, oder Hobbyspiele zu implementieren. Dies bietet sich besonders an, da die notwendige Hardware vergleichsweise günstig ist. Zur Erfassung professioneller Dart-Turniere sollte auf die Multi-Kamera Lösung, beziehungsweise auf 3D-Kameras zurückgegriffen werden.





