%!TEX root = ../thesis.tex

\chapter{Einleitung}
\label{chap:intro}

%\todo{Einleitung schreiben}
%Dartsport wachsende zahlen
%Elektronik dart vs stahldart
%Punkte Erfassung
%Einfaches erfassen
%http://www.sport1.de/darts-sport/2016/01/rekordwerte-der-darts-weltmeisterschaft-auf-sport1
%http://www.quotenmeter.de/n/90337/2015-bleibt-das-jahr-des-darts-rekords-2017-verfuenffacht-den-senderschnitt


Der Dartsport begeistert europaweit immer mehr Menschen. 2015 erreichten die Übertragungen der Darts-Weltmeisterschaft die höchsten Zuschauerzahlen in der Geschichte der Ausstrahlung des Sports \autocite{quotenmeter2017}. Für die deutsche Darts Übertragung war die kürzlich abgeschlossene Weltmeisterschaft, vom 15. Dezember 2016 bis 2. Januar 2017, eine weitere Steigerung. So schauten im Schnitt 900.000 Deutsche die einzelnen Runden der Weltmeisterschaft. Diese wurde zum 24. Mal von der "`Professional Darts Corporation"' (PDC) ausgetragen. In diesem Jahr traten insgesamt 72 Teilnehmer um den Titel an. Das Thema hat somit zum Zeitpunkt der Bachelorarbeit eine gewisse Aktualität. 
Der Gewinner des Turniers Michael van Gerwen, eine der Ikonen des Sportes, konnte am Ende ein Preisgeld von 350.000 Pfund gewinnen. Insgesamt gab es ein Preisgeld von 1.650.000 Pfund \autocite{PDC2016}.
2004 wurde zum ersten Mal Darts im Deutschen Fernsehen übetragen, damals noch auf dem Sender "`Deutsches Sport Fernsehen"' (DSF)\autocite{GaOn2016}. Seitdem steigen Beliebtheit und Zuschauerzahlen stetig.

Die Spieler haben einen strikten Trainingsplan, wie es bei Spitzensportlern in der Regel der Fall ist. So gilt es bestimmte Maßstäbe zu überwachen und zu erfassen, wie zum Beispiel den "`3 Dart"' Durchschnitt, also der Punktzahl die mit 3 Darts erzielt werden kann \autocite[98-100]{dph2015}.
Somit gilt es Trainingsziele zu erreichen und zu überwachen, wie und wann diese erreicht werden.

\section*{Motivation}
\label{sec:motivation}
%\todo{Motivation formulieren}
%\begin{itemize}
%\item Motivation
%\item Tracking von Trainings vereinfachen
%\item Möglichkeit automatischer Statistik über Abweichung Präzision und accuracy
%\item Verbesserung des Spielers
%\item Möglichkeit für automatische Spiele
%\item Keine händische Notation mehr erforderlich
%\end{itemize}
Wie bereits beschrieben, erfreut sich der Dartsport immer größerer Beliebtheit. Zudem steigen die zu erreichenden Preisgelder und die Zahl der Amateur-Turniere stetig an. Somit steigt auch der Druck auf die Spieler,  konstant gute Leistungen zu erbringen. Damit hebt sich das Wettbewerbsniveau, sodass bereits geringe Schwankungen der Leistung ausschlaggebend über Sieg und Niederlage sein können. Daher wird es immer wichtiger, einem beständigen,etablierten Trainingsplan zu folgen und diesen mit einfachen Methoden zu überwachen. 

Heute gibt es Software, in der manuell die erzielte Punktzahl eines jeden Wurfes eingetragen werden kann. Die einfachste Variante wäre dabei, ein einfaches Tabellenbearbeitungsprogramm zu nutzen, um eine Übersicht über einzelne Würfe zu behalten. 
Durch die ständigen Unterbrechungen wird der Spieler allerdings in seiner Konzentration gestört. 
Daher wäre eine automatische Erfassung der Würfe eine Erleichterung für die Spieler. Ein weiterer Punkt ist, dass aktuell auch in den im TV übertragenen Turnieren die Punktzahlen der Spieler manuell erfasst und eingetragen werden müssen und keine automatische Überprüfung dessen stattfinden kann. 

Weiterhin böte sich im Hobby Bereich die Möglichkeit andere Dartmodi und Spiele anzubieten. Ein bei Hobby Dartern verbreitetes Spiel ist beispielsweise das "`Rennen"', hier müssen nacheinander die Felder 1 bis 20 getroffen werden. Sieger wird, wer dies als Erster schafft. Zusätzlich gibt es noch viele weitere Möglichkeiten, die angeboten werden können.

Das Ziel dieser Bachelorarbeit ist es nun eine Basis für solche Anwendungen zu schaffen und die Erkennung von Darts in einem Board mit einfachen Mitteln zu ermöglichen.