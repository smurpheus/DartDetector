%!TEX root = ../thesis.tex

\chapter{Grundlagen}
\label{chap:fundamentals}
In diesem Kapitel werden Grundlagen des Dartsports und der Bildverarbeitung erläutert, welche für das Verständnis dieser Arbeit wichtig sind. 

Zum einen werden in \prettyref{sec:darts} die Grundlagen des Dartsportes erläutert, auf deren Wesen die Idee dieser Abhandlung fußt. Hierbei wird von dem Dartboard bis zu den Darts und den allgemeinen Turnierregeln ein Überblick gegeben, um den Nutzen und die Motivation verständlicher zu machen.

Weiterhin werden in \prettyref{sec:basics} grundlegende Informationen der Bildverarbeitung vermittelt, zudem wird ein Eindruck der, zur Implementierung, genutzten Bibliotheken vermittelt.

%Anschließend wird im \prettyref{sec:setup} das genutzte Testsetup dargestellt, welches für die praktische Implementierung genutzt wird.
\section{Dartsport}
\label{sec:darts}
Darts ist ein Sport, welcher eine gute Geschicklichkeit und ein besonderes Maß an Visuomotorik erfordert. Dabei wird ein Pfeil, genannt Dart, mit der Hand auf eine Zielscheibe geworfen, um Punkte zu erzielen. Damit es eine einheitliche Regulierung für das Spielmaterial gibt, wurde der Dartsport im Jahre 1920 einer ersten Standardisierung unterzogen. Die 1924 gegründete "`National Dart Association"' hat diesen zum Standard ihrer Liga erklärt \autocite[5]{guide2013}. Diese Standardisierung findet sich in vielen heutigen Regelwerken der verschiedenen Verbänden wieder; unter Anderem auch in der Sport- und Wettkampfordnung des Deutschen
Dart-Verbandes (DDV) \autocite{DartsRegel2016}. Aus dieser sind die folgenden Erklärungen und Reglungen entnommen. 

Zunächst einmal zum Dartboard. Das heutzutage genutzte Dartboard ist in Abbildung \prettyref{Fig:dartboard} zu sehen. In dieser Form wurde es zum Zeitpunkt der Standardisierung 1924 geschaffen. Die Größe des gesamten Boards und der einzelnen Felder ist hierbei normiert und standardisiert, wie es in Abbildung \prettyref{Fig:dartboard} abzulesen ist. 

Heute werden sogenannte Boards vom Typ "`Bristle"', welches sich von dem englischen Wort für "`Bürste"' ableitet, verwendet. Diese Boards werden aus Sisal-Fasern gefertigt, welche in Aufprallrichtung der Pfeile aufgestellt sind, sodass sich jene zwischen die Fasern schieben können. \autocite[6]{dph2015}. Die Drähte, welche die einzelnen Felder voneinander trennen werden "`Spider"' genannt.
\begin{figure}
\includegraphics[width=\textwidth]{media/Dartboard_Abmessungen}\\
\caption{\textbf{Standardisiertes Dartboard\cite{Board2016}}
}
\label{Fig:dartboard}
\end{figure}

Die Ziffern sind von Oben im Uhrzeigersinn wie folgt angeordnet:  20-1-18-4-13-6-10-15-2-17-3-19-7-16-8-11-14-9-12-5. Diese Art der Anordnung wird als "`London Board"' bezeichnet. Hierbei wird der äußere, 8mm breite Ring als "`Double Ring"' bezeichnet, welcher die Punktzahl des Feldes verdoppelt. Der Innere wird als "`Triple Ring"' bezeichnet und verdreifacht die Punktzahl des Feldes. Der "`Bull"', der innere Bereich des Boards, ist unterteilt in den grünen Teil, genannt "`Half Bull"', und das "`Bullseye"', diese zählen 25, beziehungsweise 50 Punkte. 
Somit ist das "`Bullseye"' entgegen landläufiger Meinung, nicht das Feld mit der höchsten Punktzahl. Es gibt sogar gleich 4 Felder, die eine höhere Punktzahl erbringen. Die Triple 20, 19, 18 und 17. Die Tripple 20 hat somit mit 60 Punkten die höchste zu erreichende Punktzahl. Im folgenden werden Triple Felder als "`TX"' bezeichnet, wobei X für die Ziffer des jeweiligen Feldes steht, zum Beispiel also T20. Eine weitere nennenswerte Schreibweise ist analog die für Doppelte Felder mit "`DX"'. 

\begin{figure}
\includegraphics[width=\textwidth]{media/Dartsfield}\\
\caption{\textbf{Seitenansicht von Board und Standleiste 
\cite[8]{DartsRegel2016}}
}
\label{Fig:dartsetup}
\end{figure}


Das Board wird mit dem Mittelpunkt des "`Bullseyes"' auf eine Höhe von 1,73 Metern gehängt, sodass sich das Feld mit der 20 mittig oben befindet \autocite[6-8]{DartsRegel2016}. In Abbildung \prettyref{Fig:dartsetup} ist eine Seitenansicht einer Dartanlage nach dem Regelwerk des DDV zu sehen. Die Standleiste, oder "`Oche"', befindet sich hier in einer Entfernung von 2.37 Metern zur Front-Seite des Boards.

Als letzten Teil der Ausrüstung gilt es den Dart selbst zu beschreiben. Bei einem Dart, wie in Abbildung \prettyref{Fig:darts} zu sehen, sind grundsätzlich vier Hauptbestandteile zu unterscheiden:
\begin{enumerate}
    \item Point, Tip oder Spitze
    \item Barrel
    \addtocounter{enumi}{1}
    \item Shaft
    \addtocounter{enumi}{1}
    \item Flight
\end{enumerate}
\begin{figure}
\includegraphics[width=\textwidth]{media/Dart}\\
\caption{\textbf{Aufbau eines Darts\cite{dart2006}}
}
\label{Fig:darts}
\end{figure}
Die weiteren Bestandteile sind optional und nicht zwingend notwendig für den Dart. 
\begin{enumerate}
	\addtocounter{enumi}{1}
	\addtocounter{enumi}{1}
    \item O-Ring, um den Shaft (4) fester im Barrel (2) zu befestigen
    \addtocounter{enumi}{1}
    \item Collar, dient dazu das Ende des Shafts zusammenzudrücken, um den Flight sicherer zu befestigen
    \addtocounter{enumi}{1}
    \item Flight Protector, dient dazu den Flight vor dem Auftreffen anderer Darts zu schützen

\end{enumerate}

Darts wird im Turniersport hauptsächlich in zwei verschiedenen Varianten gespielt, welche sich sehr ähneln. Zum einen gibt es <501> und zum anderen <301>. Für beide Varianten gilt der Modus "`Straight in"' und "`Double Out"'.
"`Straight in"' bedeutet dabei, dass mit jedem beliebigen Feld eröffnet werden kann.

Jedes Spiel, oder Match, besteht aus einer definierten Anzahl sogenannter "`Sets"'. Ein "`Set"' wiederum besteht seinerseits aus einer bestimmten Anzahl "`Legs"'.

In der Variante 501 entscheidet ein Spieler (Team) ein "`Leg"' für sich, wenn er als Erster auf genau 0 herunter spielt. Analog gilt dies für Variante 301.
Hierbei ist zu beachten, dass nicht überworfen werden darf.
Die Anzahl der dabei benötigten Würfe spielt dabei keine Rolle. Ein Wurf besteht hierbei aus drei geworfenen Darts.
 
Des weiteren gilt, dass durch den Modus "`Double Out"' das Leg mit dem Erzielen eines Doppel-Feldes zu beenden ist. Bei einem "`Straight Out"' wäre diese Einschränkung nicht gegeben.
Es ist es zusätzlich möglich ein "`Bull Out"' zu erlangen, bei dem das Leg mit dem Treffen des "`Bulls-Eyes"' beendet werden kann. 
Wird die Punktzahl durch einem Wurf, auf 1 oder in den negativen Bereich reduziert, so gilt er als überworfen und wird als "`Bust"' bezeichnet. In diesem Fall wird die Punktzahl vor dem Wurf wiederhergestellt. \autocite[5]{DartsRegel2016} 

Hat ein Spieler (Team) die definierte Anzahl an Legs gewonnen, entscheidet er ein Set für sich. Gewonnen ist ein Match von dem Spieler (Team), der als Erster die definierte Anzahl an Sets gewinnt.

\begin{example*}[Ein Spieler hat 10 Punkte verbleibend] 

Es müsste eine D5 erzielt werden, um mit dem ersten Dart abzuschließen. Es folgen verschiedene Szenarien um die Regeln zu veranschaulichen.
\begin{enumerate}
	\item Es wird eine D5 geworfen, der Spieler gewinnt somit das Leg
	\item Es wird eine 8 erzielt. Der zu tilgende Punktestand beträgt somit 2 und es müsste eine D1 erzielt werden, um das Leg zu beenden. Der Spieler trifft im Anschluss die D1 und entscheidet das Leg somit für sich. 
	\item Es wird eine 8 erzielt. Der zu tilgende Punktestand beträgt somit 2. Statt der D1 wird eine einfache 1 erzielt und somit ist der Wurf ungültig und die Punktzahl wird auf 10 zurück gesetzt.
\end{enumerate}
\end{example*}

Rechnerisch ist es bei 501 also möglich, mit 9 Darts zu beenden (3xT20, 3xT20, 2xT20+Bullseye), der sogenannte "`Neun Darter"'.

Die bisher beschriebene Dart Ausstattung wird als "`Steeldart"' bezeichnet, da die Darts in der Regel eine Stahlspitze besitzen und einen Metallschaft haben. Dem gegenüber steht das sogenannte "`Softdart"'. Beim Softfdart besitzen die Darts eine Kunststoffspitze statt einer Stahlspitze. Dem entsprechend ist auch das Board ein Kunststoffboard mit vorgefertigten Löchern, in welche die Spitzen der Pfeile eindringen können. 

\section{Grundlagen Bildverarbeitung}
\label{sec:basics}
Im folgenden werden einige Grundlagen beschrieben, die nötig sind, um die verwendeten Algorithmen und Vorgehensweisen nachvollziehen zu können. Es wird davon ausgegangen, dass grundlegende Kenntnisse zum Aufbau eines digitalisierten Bildes vorhanden sind. 
\subsection{Physik und Geometrie einer Kamera}
\label{sec:camerageo}
Dieser Abschnitt beginnt mit der Entstehung der verarbeiteten Bilder. Ein Verständnis hierfür ist Voraussetzung, um die Abbildung von 3D-Objekt auf die 2D-Bilder nachvollziehen zu können.

Zur Vereinfachung wird vorerst von dem sogenannten "`Pinhole Camera Model"' ausgegangen. In Abbildung \prettyref{Fig:simple-pinhole} ist ein solches Modell dargestellt. 

Mittig ist die Barriere zu sehen, in welcher sich ein Loch c befindet, durch das die Lichtstrahlen auf die Image Plane I treffen können. Daraus resultiert ein auf dem Kopf stehendes Bild. $c$ ist hierbei das Zentrum des Koordinatensystems.
Die Distanz zwischen I und der Barriere ist bei heutigen Kameras die Focal Length $(f)$. Hierbei gilt es zu bemerken, dass Kameras heutzutage nicht mehr nur ein Loch besitzen, sondern eine Linse eingesetzt ist, welche die Lichtstrahlen bricht und auf die Image Plane lenkt. Das grundlegende Modell bleibt aber auch bei heutigen Kameras bestehen. Wie die Linse das Bild verändert wird später erläutert. 

Auf diese Weise entsteht eine Abbildung des Punktes $P$ auf der zwei dimensionalen Fläche $I$. Sind alle Parameter bekannt, lässt sich folglich bestimmen, wo der Punkt $P$ später im Bild zu sehen sein wird. 

\begin{figure}
\includegraphics[width=\textwidth]{media/simple-pinhole.png}\\
\caption{\textbf{Einfache Darstellung der Funktionsweise einer einfachen Loch Kamera}
}
\label{Fig:simple-pinhole}
\includegraphics[width=\textwidth]{media/pinhole_camera_model}\\
\caption{\textbf{Pinhole Camera Modell \autocite{OpencvCamera2016}.}
}
\label{Fig:pinhole}
\end{figure}

In Abbildung \prettyref{Fig:pinhole} ist eine andere Betrachtungsweise zu sehen. Hier wurde die Image Plane rotiert und vor den Koordinatenursprung verschoben. Zu sehen sind die Achsen u und v, welche der X- und Y-Achse des Bildes entsprechen. Rot markiert ist zu sehen wie sich der Objekt- oder Welt-Punkt auf der Image Plane am Punkt $(u, v)$ abbildet. Auf der Image Plane befindet sich in heutigen Kameras ein Bildsensor, der die analogen Lichtsignale zu einem digitalen Bild umwandelt. 

Entlang der Z-Achse des Kamera-Koordinatensystems befindet sich die optische Achse. Der Punkt $(c_x,c_y)$, in dem die optische Achse die Image Plane schneidet, ist der "`principal point"' oder das "`Centre"' \autocite[8]{Medioni:2004:ETC:993884}

Nun gilt es eine mathematische Abbildung zu finden, um den Objektpunkt in den entsprechenden Bildpunkt zu transformieren. Folglich muss eine Rotation und eine Translation vorgenommen werden:
Die Rotationsmatrix mit dem Namen $R$ und die Translationsmatrix $t$. Diese bilden die extrinsischen Parameter der Kamera, welche bestimmen, wo diese sich im Raum im Verhältnis zu einem Objekt befindet. Anhand dieser Parameter kann ein  Objektpunkt $P = \begin{bmatrix}X & Y& Z \\\end{bmatrix}^\intercal$ in den Punkt $ k = \begin{bmatrix}x & y& z \\\end{bmatrix}^\intercal$ des Kamera-Koordinatensystems übertragen werden. Somit gilt: 
$k = R * P + t$ 

Nun gilt es den Punkt im Kamera-Koordinatensystem zum tatsächlichen Bildpunkt $m=(u,v)$ zu übertragen. Hierfür müssen die intrinsischen Parameter der Kamera bestimmt werden. Diese bleiben bei einer spezifischen Kamera bestehen, vorausgesetzt die Linse ist starr und nicht fokussierbar, ansonsten müssen sie erneut bestimmt werden.
Zu diesen Parametern zählen die bereits erwähnte focal length und das Centre der Kamera. Wie vorher beschrieben wirkt sich die Linse der Kamera auf das Bild aus. Die Linse verursacht eine Verzerrung des Bildes. Zum Einen eine radiale Verzerrung (auch als "`Fish Eye"' bekannt), ein Beispiel zur Verdeutlichung ist in Abbildung \prettyref{Fig:radialdistortion} zu erkennen, zum Anderen findet eine Verschiebung statt (tangential disortion).

\begin{figure}
\includegraphics[width=\textwidth]{media/distortion_examples}\\
\caption{\textbf{Radiale Verzerrung Beispiel \autocite{OpencvCamera2016}}
}
\label{Fig:radialdistortion}
\end{figure}

Um nun den tatsächlichen Bildpunkt zu erhalten, müssen also noch die Verzerrungs-Parameter berücksichtigt werden. Damit ergibt sich:
$sk = A [R | t] * P 
$
wobei $A = \begin{bmatrix} f_x & \gamma & c_x \\
                             0 &  f_y   & c_y \\
                             0 &  0     &  1\end{bmatrix}$ und $s$ ein von der Auflösung abhängiger Skalierungsfaktor ist. $\gamma$ gibt an, in welchem Winkel die Image Plane zum Kamera-Koordinatensystem rotiert ist. Zuletzt wird nun der eigentliche Bildpunkt, beziehungsweise Pixel, errechnet, indem $m=(x/z, y/z)=(u,v)$.



\subsection{Programmierung von Computer Vision}
\label{sec:opencv}
Da es viele Basisprobleme und Anwendungen in der Computer Vision gibt, wurde eine open source Library implementiert. Die geläufigste ist die Open Source Computer Vision Library, kurz OpenCV, sie wurde unter BSD Lizenz veröffentlicht, sodass sie nahezu ohne Einschränkungen genutzt werden kann. Die Bibliothek ist größtenteils in C\# und C++  geschrieben und wurde anfänglich von Intel entwickelt \textbf{\autocite[512--]{Medioni:2004:ETC:993884}}. Einige Gründe hierfür waren der Wunsch, der Forschungsgemeinschaft zu ermöglichen neue Anwendungen implementieren zu können, ohne sich um die grundlegende Implementierung kümmern zu müssen. So bietet OpenCV auch Möglichkeiten um eine Kamera mit einfachen Mitteln kalibrieren zu können. Auf das Vorgehen dabei wird in \prettyref{sec:camera} eingegangen.
Die Bibliothek ist auf allen gängigen Betriebssystemen verfügbar, somit existiert je eine Version für Windows, Linux und Mac. Zudem gibt es für mobile Applikationen noch eine Unterstützung von Android.

Für OpenCV sind Interfaces zu anderen Sprachen implementiert worden, so ist es möglich OpenCV in Java und Python zu nutzen \autocite{OpenCV2016}. Insgesamt werden rund 2500 optimierte Algorithmen angeboten, hierzu zählen Algorithmen im Bereich Machine Learning, 3D Objekt Erkennung, Gesichtserkennung und vielem mehr. Zudem befinden sich Interfaces zu Nvidias CUDA, eine Bibliothek zur parallelen Berechnung \autocite{cuda2017}, und zu OpenCL aktuell in der aktiven Entwicklung.




